\title{Maya Soft Body Deformer Plugin Using Shape Matching \\ \large SFX - Tricks of the trade: Project Report}

\author{\begin{tabular}{ccc}
    Isabell Jansson & Ronja Grosz  & Jonathan Bosson \\
    \small isaja187@student.liu.se &\small rongr946@student.liu.se &\small jonbo665@student.liu.se \\
\end{tabular}}
\date{\today}

\documentclass[12pt, twocolumn]{article}
\usepackage{graphicx} % Figures
\usepackage{amsmath}

\begin{document}
\twocolumn[
\begin{@twocolumnfalse}
\maketitle

\begin{abstract}

\end{abstract}

\end{@twocolumnfalse}
]

\section{Introduction}
    Soft body deformation is a computer graphics method for simulating the deformation of a soft body. 
    There exists many techniques for achieving soft body deformation.
    In this paper we present the method and results from implementing a soft body deformer plugin for the 3D computer graphics software Maya. The implementation is based on the meshless shape matching method proposed by M\"uller et al.~\cite{shapematching}.

\section{Background}
    Soft deformable bodies is a field in computer graphics that focuses on the physical and visual simulation of the motion and characteristics of a soft deformable body.
    Soft deformable bodies are objects whose bodies can change by being deformed without losing its characteristic shape.
    Several different methods have been proposed to simulate the deformation of soft deformable bodies.
    The spring mass model is one of the simpler methods where the soft body is approximated by a set of masses linked by springs.
    The method is popular for simulating the deformation of cloth.
    Two other methods are the finite element method and the energy minimization method.

    The shape matching approach proposed by M\"uller et al.~\cite{shapematching} is a meshless shape matching technique for deformation of soft bodies.
    It is a meshless technique since the object is only represented by a point cloud.
    There is no need for connectivity information between the points since the shape matching keeps the form of the object.
    This is done by driving the deformed points towards the position of the points of the original shape, see figure~\ref{fig:def}.
    
    \begin{figure}
    \includegraphics[width=\linewidth]{img/deformation2.png}
    \caption{The shape matching process where the original shape $\mathbf{x}^0_i$ is matched to the deformed shape $\mathbf{x}_i$ and pulled towards the goal shape $\mathbf{g}_i$. Image source:~\cite{shapematching}.}
    \label{fig:def}
    \end{figure}
    
\section{Shape matching implementation}
    The soft body is represented by a point cloud, where each particle has a mass $m_i$, an initial position $\mathbf{x}_i^0$ representing the shape of the object and a deformed position $\mathbf{x}_i$.
    The soft body is affected by external forces such as gravity, collision impulse and friction impulse caused by a collision with an object.
    The external forces $f_{ext} = F_g + J_c + J_f$ is what causes the deformation of the object.
    The new velocities and positions for the particles are calculated using the external forces and Euler integration.
    The collision impulse $J_c$ is calculated through Equation~\ref{eq:col} where $\epsilon$ is the elasticity of the soft body, $n$ is the contact normal and $\Delta{v}$ is the velocity difference between the soft body and the static object it collides with.
    The friction impulse is calculated through Equation~\ref{eq:fric}, where $f$ is the friction.

    \begin{equation} \label{eq:col}
        J_c = -(1 + \epsilon)(n(n\cdot{\Delta{v}}))m_i
    \end{equation}

    \begin{equation} \label{eq:fric}
        J_f = -f(\Delta{v} - n(n\cdot{\Delta{v}})m_i
    \end{equation}

    The shape matching is made by finding a goal position $g_i$ for every particle.
    The goal position is used to find the new velocities and positions for the particles for each time step, driving the soft body to take its original shape at $\mathbf{x}^0$.
    The new velocities and positions is calculated through Euler integration where $\alpha$ is the bounciness and stiffness of the soft body and $h$ is the time step, see Equation~\ref{eq:vel} and~\ref{eq:pos}.

    \begin{equation} \label{eq:vel}
        \mathbf{v}_i(t + h) = \mathbf{v}_i(t) + \alpha{\frac{\mathbf{g}_i(t) - \mathbf{x}_i(t)}{h}} + hf_{ext}(t)/m_i
    \end{equation}

    \begin{equation} \label{eq:pos}
        \mathbf{x}_i(t + h) = \mathbf{x}_i(t) + h\mathbf{v}_i(t + h) 
    \end{equation}

    The goal position is found through the rotation matrix $\mathbf{R}$, the initial position $\mathbf{x}^0_i$, the center of mass for the initial position $\mathbf{x}^0_{cm}$ and the center of mass for the current deformed shape $\mathbf{x}_{cm}$, see Equation~\ref{eq:goal}.
   
    \begin{equation}\label{eq:goal}
        \mathbf{g}_i = \mathbf{R}(\mathbf{x}^0_i - \mathbf{x}^0_{cm}) + \mathbf{x}_{cm}
    \end{equation}

    The rotation matrix is approximated through the linear transformation $\mathbf{A}$, see Equation~\ref{eq:A}.
    The linear transformation is composed of the relative locations $\mathbf{q}_i = \mathbf{x}^0_i - \mathbf{x}^0_{cm}$ and $\mathbf{p}_i = \mathbf{x}_i - \mathbf{x}_{cm}$, see Equation~\ref{eq:com1} and~\ref{eq:com2}.
    The rotation $\mathbf{R}$ is the rotational part of $\mathbf{A}_{pq}$ which is found through singular value decomposition.

    \begin{equation} \label{eq:A}
        \mathbf{A} = (\sum_i{m_i\mathbf{p}_i\mathbf{q}_i^{\mathbf{T}}})
        (\sum_i{m_i\mathbf{q}_i\mathbf{q}_i^{\mathbf{T}}})^{-1} 
        = \mathbf{A}_{pq}\mathbf{A}_{qq}
    \end{equation}

    \begin{equation} \label{eq:com1}
        \mathbf{x}^0_{cm} = \frac{\sum_i{m_i\mathbf{x}_i^0}}{\sum_i{m_i}}
    \end{equation}

    \begin{equation} \label{eq:com2}
        \mathbf{x}_{cm} = \frac{\sum_i{m_i\mathbf{x}_i}}{\sum_i{m_i}}
    \end{equation}

    \paragraph{Rigid deformation}\mbox{}\\[5px]
    For a rigid deformation $\alpha$ is set to one, moving the points to the goal position $\mathbf{g}_i$ exactly for each time step.
    The goal position represents a rotated and translated version of the initial shape.

    \paragraph{Linear deformation}\mbox{}\\[5px]
    For linear deformation, the linear transformation $\mathbf{A}$ in combination with the rotation matrix $\mathbf{R}$ according to Equation~\ref{eq:T} is used instead of the rotational matrix to calculate the goal position, see Equation~\ref{eq:goal}.
    The linear deformation $\mathbf{A}$ is divided by $\sqrt[3]{det(\mathbf{A})}$ to keep the volume conserved.

    \begin{equation} \label{eq:T}
        \beta\mathbf{A} + (1 - \beta)\mathbf{R}
    \end{equation}

    \paragraph{Quadratic deformation}\mbox{}\\[5px]
    To extend the motion to twisting and bending a quadratic transformation is applied.
    $\mathbf{q}$ is extended to $\mathbf{\tilde{q}} = [\mathbf{q}_x, \mathbf{q}_y, \mathbf{q}_z, \mathbf{q}_x^2, \mathbf{q}_y^2, \mathbf{q}_z^2, \mathbf{q}_x\mathbf{q}_y, \mathbf{q}_y\mathbf{q}_z, \mathbf{q}_z\mathbf{q}_x]^\mathbf{T}$.
    With the new $\mathbf{\tilde{q}}$, a new linear transformation $\mathbf{\tilde{A}}$ can be calculated the same way as Equation~\ref{eq:A}.
    To keep the volume, $\mathbf{\tilde{A}}$ is divided by $\sqrt[9]{det(\mathbf{\tilde{A}})}$.
    Together with the rotation matrix $\mathbf{\tilde{R}} = [\mathbf{R~0~0}]$, the goal position is calculated according to Equation~\ref{eq:newGoal}.

    \begin{equation} \label{eq:newGoal}
        \mathbf{g}_i = (\beta\mathbf{\tilde{A}} + (1 - \beta)\mathbf{\tilde{R}})\mathbf{\tilde{q}}
    \end{equation}
    

\section{Conclusion}

\bibliographystyle{ieeetr}
\pagebreak
\bibliography{./refs}

\end{document}
